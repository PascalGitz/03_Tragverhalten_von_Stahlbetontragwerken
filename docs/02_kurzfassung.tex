\chapter*{Kurzfassung}

In dieser Arbeit wird eine Tragwerksanalyse des Bemessungsbeispiels aus
der SIA Dokumentation D0192, Kapitel 4.2.1.1 durchgeführt. Dabei wird
ein linear-elastisches Werkstoffverhalten vorausgesetzt. Weiter wird die
Wand nur für den Grenzzustand der Tragsicherheit Typ 2 bemessen. Die
Gebrauchstauglichkeit sowie der gerissene Zustand werden in dieser
Arbeit nicht betrachtet.

In dieser Arbeit wird eine Tragwerksanalyse des Bemessungsbeispiels aus
der SIA Dokumentation D0192, Kapitel 4.2.1.1 durchgeführt. Dabei wird
ein linear-elastisches Werkstoffverhalten vorausgesetzt. Weiter wird die
Wand nur für den Grenzzustand der Tragsicherheit Typ 2 bemessen. Die
Gebrauchstauglichkeit sowie der gerissene Zustand werden in dieser
Arbeit nicht betrachtet.
In dieser Arbeit wird eine Tragwerksanalyse des Bemessungsbeispiels aus
der SIA Dokumentation D0192, Kapitel 4.2.1.1 durchgeführt. Dabei wird
ein linear-elastisches Werkstoffverhalten vorausgesetzt. Weiter wird die
Wand nur für den Grenzzustand der Tragsicherheit Typ 2 bemessen. Die
Gebrauchstauglichkeit sowie der gerissene Zustand werden in dieser
Arbeit nicht betrachtet.
In dieser Arbeit wird eine Tragwerksanalyse des Bemessungsbeispiels aus
der SIA Dokumentation D0192, Kapitel 4.2.1.1 durchgeführt. Dabei wird
ein linear-elastisches Werkstoffverhalten vorausgesetzt. Weiter wird die
Wand nur für den Grenzzustand der Tragsicherheit Typ 2 bemessen. Die
Gebrauchstauglichkeit sowie der gerissene Zustand werden in dieser
Arbeit nicht betrachtet.
In dieser Arbeit wird eine Tragwerksanalyse des Bemessungsbeispiels aus
der SIA Dokumentation D0192, Kapitel 4.2.1.1 durchgeführt. Dabei wird
ein linear-elastisches Werkstoffverhalten vorausgesetzt. Weiter wird die
Wand nur für den Grenzzustand der Tragsicherheit Typ 2 bemessen. Die
Gebrauchstauglichkeit sowie der gerissene Zustand werden in dieser
Arbeit nicht betrachtet.
In dieser Arbeit wird eine Tragwerksanalyse des Bemessungsbeispiels aus
der SIA Dokumentation D0192, Kapitel 4.2.1.1 durchgeführt. Dabei wird
ein linear-elastisches Werkstoffverhalten vorausgesetzt. Weiter wird die
Wand nur für den Grenzzustand der Tragsicherheit Typ 2 bemessen. Die
Gebrauchstauglichkeit sowie der gerissene Zustand werden in dieser
Arbeit nicht betrachtet.
In dieser Arbeit wird eine Tragwerksanalyse des Bemessungsbeispiels aus
der SIA Dokumentation D0192, Kapitel 4.2.1.1 durchgeführt. Dabei wird
ein linear-elastisches Werkstoffverhalten vorausgesetzt. Weiter wird die
Wand nur für den Grenzzustand der Tragsicherheit Typ 2 bemessen. Die
Gebrauchstauglichkeit sowie der gerissene Zustand werden in dieser
Arbeit nicht betrachtet.
In dieser Arbeit wird eine Tragwerksanalyse des Bemessungsbeispiels aus
der SIA Dokumentation D0192, Kapitel 4.2.1.1 durchgeführt. Dabei wird
ein linear-elastisches Werkstoffverhalten vorausgesetzt. Weiter wird die
Wand nur für den Grenzzustand der Tragsicherheit Typ 2 bemessen. Die
Gebrauchstauglichkeit sowie der gerissene Zustand werden in dieser
Arbeit nicht betrachtet.
