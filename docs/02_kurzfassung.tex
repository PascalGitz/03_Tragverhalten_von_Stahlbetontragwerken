\chapter*{Kurzfassung}

Diese Arbeit untersucht das Tragverhalten von Stahlbetontragwerken und stellt verschiedene Modelle und Methoden vor, die zur Beschreibung und Berechnung der Verformungen verwendet werden können. Wir werden die Grundlagen der Kontinua-, Mohrschen Analogie-, Zuggurt- und Fachwerkmodelle erläutern und sie auf zwei Versuchsexperimente anwenden: einen Dreipunktbiegeversuch und einen Vierpunktbiegeversuch. Die Ergebnisse der verschiedenen Modelle werden mit den gemessenen Verformungen aus den Versuchen verglichen und die Vor- und Nachteile sowie die Anwendbarkeit der Modelle diskutiert. Es wird gezeigt, dass die Modelle unterschiedliche Genauigkeiten und Komplexitäten aufweisen und dass die Wahl des geeigneten Modells von den Randbedingungen, der Bewehrungsführung, der Querkraftbewehrung und der Zugversteifung abhängt.