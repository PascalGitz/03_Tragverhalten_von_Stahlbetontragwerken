\chapter*{Kurzfassung}

Die vorliegende Arbeit befasst sich mit Verformungen im Stahlbetonbau. Speziell mit den Verformungen von Stabtragwerken. Dazu sind unterschiedliche Modelle zur rechnerischen Bestimmung aufgezeigt. Als übergeordnetes Ziel gilt es praxistaugliche Berechnungsmethoden zu verwenden. Dies bedeutet möglichst geringen Berechnungsaufwand bei gleichzeitig hoher Genauigkeit.  

Das einleitende Kapitel beschreibt die Hintergründe der Modelle. Es wird auf das Modell des reinen Biegeträgers, die Methode der Mohr'schen Analogie, eine Abschätzung nach der Schweizerischen Betonnorm, das Zuggurtmodell, eine Integrationsmethode zur Berücksichtigung einer nicht-linearen Momenten-Krümmungs-Beziehung, sowie abschliessend auf eine Fachwerksanalyse eingegangen. Neben den theoretischen Grundlagen zeigt dir Arbeit die Anwendung der Modelle an einem Dreipunktbiegeversuch und einem Vierpunktbiegeversuch. Die Versuchsanwendung wird jeweils mit einer ausführlichen Diskussion der Ergebnisse abgeschlossen, verifiziert an den gemessenen Versuchsdaten. 

Der Modellvergleich zeigt, dass mit Berechnungsmethoden mit konstanten Biegesteifigkeiten, wie in der Praxis üblich, Verformungen nur bedingt präzise berechnet werden können. Die Verwendung von einer nicht-linearen Momenten-Krümmungs-Beziehungen ist für die Nachrechnung der Verformungen von Versuchen, belastet bis zum Versagen, unerlässlich. Die Fachwerksanalyse liefert bei beiden Versuchen die treffendsten Ergebnisse, sofern die Fachwerkshöhe exakt bestimmt werden kann. 


Im abschliessenden Kapitel wird das Ziel beschrieben, in einer folgenden Arbeit, die aufgezeigten Modelle auf Plattentragwerke zu erweitern. Als Ansatz dazu soll die Modellierung als Trägerrost dienen.