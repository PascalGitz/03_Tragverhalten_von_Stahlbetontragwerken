\chapter*{Kurzfassung}

Die vorliegende Arbeit befasst sich mit Verformungen im Stahlbetonbau; speziell mit den Verformungen von Stabtragwerken. Dazu sind unterschiedliche Modellvorstellungen zur rechnerischen Bestimmung aufgezeigt. Als übergeordnetes Ziel gilt es praxistaugliche Berechnungsmethoden zu verwenden. Dies bedeutet möglichst geringen Berechnungsaufwand bei gleichzeitig vernünftiger Genauigkeit.  

Das einleitende Kapitel beschreibt die Hintergründe der Modellvorstellungen. Es wird auf die Modellvorstellungen des reinen Biegeträgers, die Methode der Mohr'schen Analogie, eine Abschätzung nach der Schweizerischen Betonnorm, das Zuggurtmodell, eine Integrationsmethode zur Berücksichtigung einer nicht-linearen Momenten-Krümmungs-Beziehung sowie abschliessend auf eine Analyse mit Fachwerksmodellen eingegangen. Neben den theoretischen Grundlagen zeigt die Arbeit die Anwendung der Modellvorstellungen an einem Dreipunktbiegeversuch und einem Vierpunktbiegeversuch. Diese wird jeweils mit einer ausführlichen Diskussion der Ergebnisse abgeschlossen, verifiziert an den gemessenen Versuchsdaten. 

Der Vergleich der Modellvorstellungen zeigt, dass mit Berechnungsmethoden mit konstanten Biegesteifigkeiten, wie in der Praxis üblich, Verformungen nur bedingt präzise berechnet werden können. Die Verwendung von einer nicht-linearen Momenten-Krümmungs-Beziehungen ist für die Nachrechnung der Verformungen von Versuchen, belastet bis zum Versagen, unerlässlich. Die Analyse mit Fachwerksmodellen liefert bei beiden Versuchen die treffendsten Ergebnisse, sofern der Hebelarm der inneren Kräfte bestimmt werden kann. 


Im abschliessenden Kapitel wird das Ziel beschrieben, in einer folgenden Arbeit, die aufgezeigten Modellvorstellungen auf Plattentragwerke zu erweitern. Als Ansatz dazu soll die Modellierung als Trägerrost dienen.